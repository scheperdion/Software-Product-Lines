\documentclass[]{article}

%opening
\title{Severe Weather Alert System}
\author{Dion Scheper, Terence Beijloos, Bart van Nimwegen}

\begin{document}

\maketitle

\section{Context/Background}
For the course software product lines we are developing a software product line. 

\section{Motivation}
% why is it a good idea to develop a product line / conduct your proposed project etc.)?
Creating a software product line for a severe weather app is a compelling idea because the domain offers significant variability, with many weather-related features like temperature, precipitation, and wind that can be tailored to different user needs. 
Developing a minimum viable product is relatively straightforward, as the app could start with core functionalities like basic alerts for severe weather and expand later. Additionally, the project offers an opportunity to tackle the optional feature problem, where weather properties may seem independent but could have hidden dependencies during implementation. 
Finally, the app isn't heavily reliant on external services; weather data can easily be mocked during development, making it easier to test and iterate.

\section{Methodology}
% This question can entail a software development methodology (used processes for working in your team, tools etc.) and/or research methodology (empirical research, design science research…)
\begin{itemize}
	\item agile / remote
	\item git / github for code sharing
	\item featureIDE
	\item optional: include aspectJ if possible
\end{itemize}

\section{Deliverables}
%For example, what features do you plan to implement, what kind of variability mechanisms will be used and why, etc.
\begin{itemize}
	\item 
\end{itemize}


\section{Area's of responsibility}


\end{document}
